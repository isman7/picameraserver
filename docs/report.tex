% Definimos el estilo del documento
\documentclass[11pt,a4paper,spanish]{article}
% Utilizamos el paquete para utilizar español
\usepackage[spanish]{babel}
% Utilizamos un paquete para gestionar los acentos y las eñes
\usepackage[utf8]{inputenc}
% Utilizamos el paquete para gestionar imagenes jpg
\usepackage[pdftex]{graphicx}
% Definimos la zona de la pagina ocupada por el texto
\usepackage{psfrag}

\usepackage{hyperref}

\usepackage{float}

\oddsidemargin -1.0cm
\headsep -2.4cm
\textwidth=18.5cm
\textheight=26 cm
\usepackage{fullpage}
\usepackage{fullpage}
\usepackage{listings}
\usepackage{color}


\usepackage[all]{xy}


\usepackage{fancyhdr}


\usepackage{lscape}

\usepackage{charter}

\usepackage{multicol}

\usepackage{framed}

\begin{document}



\title{{\large Enginyeria Electrònica de Telecomunicacions\\ \texttt{Xarxes de Comunicacions}} \\ Servidor MJPEG para Raspberry Pi }
\author{ Ismael Benito Altamirano}
%\date{13 de maig de 2013}
\maketitle

\pagestyle{fancy}


\renewcommand{\headrulewidth}{0.4pt}
\renewcommand{\footrulewidth}{0.4pt}

\lhead{Xarxes de \\ Comunicacions}
\chead{}
\rhead{Servidor MJPEG para Raspberry Pi}
\lfoot{Ismael Benito Altamirano}
\rfoot{\thepage}
\cfoot{}
\renewcommand{\headrulewidth}{0.4pt}
\renewcommand{\footrulewidth}{0.4pt}

\fontsize{10pt}{10pt} \selectfont 

\begin{abstract}
\end{abstract}


\definecolor{dkgreen}{rgb}{0,0.6,0}
\definecolor{gray}{rgb}{0.5,0.5,0.5}
\definecolor{mauve}{rgb}{0.58,0,0.82}
 
\lstset{ %
  language=Python,                % the language of the code
  basicstyle=\footnotesize,           % the size of the fonts that are used for the code
  numbers=left,                   % where to put the line-numbers
  numberstyle=\footnotesize,          % the size of the fonts that are used for the line-numbers
  stepnumber=2,                   % the step between two line-numbers. If it's 1, each line 
                                  % will be numbered
  numbersep=5pt,                  % how far the line-numbers are from the code
  backgroundcolor=\color{white},      % choose the background color. You must add \usepackage{color}
  showspaces=false,               % show spaces adding particular underscores
  showstringspaces=false,         % underline spaces within strings
  showtabs=false,                 % show tabs within strings adding particular underscores
  frame=single,                   % adds a frame around the code
  tabsize=2,                      % sets default tabsize to 2 spaces
  captionpos=b,                   % sets the caption-position to bottom
  breaklines=true,                % sets automatic line breaking
  breakatwhitespace=false,        % sets if automatic breaks should only happen at whitespace
  title=\lstname,                   % show the filename of files included with \lstinputlisting;
                                  % also try caption instead of title
  numberstyle=\tiny\color{gray},        % line number style
  keywordstyle=\color{blue},          % keyword style
  commentstyle=\color{dkgreen},       % comment style
  stringstyle=\color{mauve},         % string literal style
  escapeinside={\%*}{*)},            % if you want to add a comment within your code
  morekeywords={*,...}               % if you want to add more keywords to the set
}




\end{document}

%\begin{figure}[h!]
%\centering	\includegraphics[width=0.6\textwidth]{pitagores.jpg} 
%\caption{Teorema de Pitàgores}
%\end{figure}



%\begin{figure}[h!]
%\centering	\includegraphics[trim=3cm 8cm 3cm 8cm, clip, width=0.6\textwidth]{MATLAB_1.pdf}
%\caption{Resposta frequencial d'un filtre FIR(esquerra) i un filtre IIR(dreta).}
%\label{fig:fir_iir}
%\end{figure} 
